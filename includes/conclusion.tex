\indent Podľa zadaných cielov tejto práce sme zanalyzovali teraz dostupné a populárne co-working aplikácie. Na základe tejto analýzy sme zistili nedostatky a problémy tychto splikácii a prišli s návrhom vlastnej aplikácie. Následne sme spravili prieskum aktuálnych trendov a technológii používaných pri tvorbe web aplikácii a vybrali sme technológie, ktoré sme neskôr použili na implementovanie našej aplikácie. Ďaľším krokom, ktorý sme splnili bolo navrhnúť celú aplikáciu s prihliadnutím na technológie, ktoré sme sa arozhodli použiť. Následne sme podla návrhu aplikáciu implementovali ako webovú aplikáciu. Podla cieľa, ktorý sme si stanovli aby aplikácia bola implementovaná formou desktopovej aplikácie sme webovú aplikáciu exportovali na desktopovú. Po tomto kroku sme aplikáciu podrobili testu na fukcionalitu, či aplikácia spĺňa fukcionalitu, ktorú sme jej stanovili. Z výsledku týchto testov môžeme skonštatovať, že aplikácia spĺňa nami stanovené kritériá.

\indent Aplikácia umožňuje spolupracovanie a komunikáciu skupín ludí pracujúcich na dosiahnutí nimi stanoveného ciela. Cez aplikáciu je možné komunikovať v uzavretej skupine ludí a manažovať túto skupinu podľa potreby.

\indent Cieľom tejto aplikácie je zjednodušiť komunikáciu a spoluprácu tímov pracujúcich na projektoch či už v rámci školy alebo aj mimo nej. Jej hlavným cieľom je mať pod jedným účtom možnosti, na ktoré v iných aplikáciách treba mať viaceho účtov a je nutné niektoré možnosti importovať ako doplnkové služby. Ďaľším cielom aplikácie je priniesť študentom bezplatnú aplikáciu na spolupracovanie a komunikáciu nakoľko väčšina iných aplikácii si vyžaduje poplatky buď za celkové využívanie aplikácie alebo za využívanie jej častí prípadne pridanie doplnkov. 

\indent Práca na aplikácii tu zdaleka nemusí končiť. Aplikácia bola navrhovaná aj implementovaná tak aby sa ďalej dala rozširovať a dali sa do nej dopĺnať ďaľšie funkcie a možnosti. Možnými rozšíreniami, ktoré je možne ľahko implementovať pri pokračovaní sú napríklad notifikácie v aplikácii, pridanie možnosti posielať emoticony, fotky, obrázky, možnosť implementovať prevzatie udalostí z kalendárov iných aplikácii ako je napríklad Google kalendár a iné. Aplikácia môže veľmi dobre poslúžiť ako odrazový mostík iným študentom pri záujme pokračovať vo vývoji tejto aplikácie alebo pri tvorbe vlastnej, podobnej aplikácie. 